% Template:     Template Controles LaTeX
% Documento:    Configuraciones del template
% Versión:      1.1.8 (03/09/2017)
% Codificación: UTF-8
%
% Autor: Pablo Pizarro R.
%        Facultad de Ciencias Físicas y Matemáticas
%        Universidad de Chile
%        pablo.pizarro@ing.uchile.cl, ppizarror.com
%
% Sitio web:    [http://latex.ppizarror.com/Template-Controles/]
% Licencia MIT: [https://opensource.org/licenses/MIT]

% CONFIGURACIONES GENERALES
\def\defaultinterline {1.0}       % Interlineado por defecto [pt]
\def\defaultnewlinesize {11.0}    % Tamaño del salto de línea [pt]
\def\documentlang {es-CL}         % Define el idioma del documento
\def\fontdocument {lmodern}       % Tipografía (lmodern,arial,helvet)
\def\numberedequation {true}      % Ecuaciones con \insert... numeradas
\def\pointdecimal {false}         % Decimales con punto en vez de coma
\def\tablepadding {1.0}           % Ancho de celda de las tablas

% MÁRGENES DE PÁGINA
\def\pagemarginbottom {2.7}       % Margen inferior página [cm]
\def\pagemarginleft {2.54}        % Margen izquierdo página [cm]
\def\pagemarginright {2.54}       % Margen derecho página [cm]
\def\pagemargintop {2.30}         % Margen superior página [cm]

% CONFIGURACIÓN DE LAS LEYENDAS - CAPTION
\def\captionlessmarginimage {0.1} % Margen sup/inf de figuras si no hay leyenda [cm]
\def\captionlrmargin {2.0}        % Márgenes izq/der de la leyenda [cm]
\def\captionalignment {justified} % Alineación leyenda (justified,centered,left,right)
\def\captiontbmarginfigure {9.35} % Margen sup/inf de la leyenda en figuras [pt]
\def\captiontbmargintable {7.0}   % Margen sup/inf de la leyenda en tablas [pt]
\def\captiontextbold {false}      % Etiquetas (Código,Figura,Tabla) en negrita
\def\codecaptiontop {true}        % Leyenda arriba del código fuente
\def\figurecaptiontop {false}     % Leyenda arriba de las imágenes
\def\showsectiononcaption {false} % Muestra el número de sección en las leyendas
\def\tablecaptiontop {true}       % Leyenda arriba de las tablas

% CONFIGURACIÓN DE LOS COLORES DEL DOCUMENTO
\def\captioncolor {black}         % Color de la etiqueta (Código,Figura,Tabla)
\def\captiontextcolor {black}     % Color de la leyenda
\def\colorpage {white}            % Color de la página
\def\highlightcolor {yellow}      % Color del subrayado con \hl
\def\linkcolor {black}            % Color de los links del doc.
\def\maintextcolor {black}        % Color principal del texto
\def\numcitecolor {black}         % Color del número de las referencias o citas
\def\showborderonlinks {false}    % Color de un links por un recuadro de color
\def\subsubtitlecolor {black}     % Color de los sub-subtítulos
\def\subtitlecolor {black}        % Color de los subtítulos
\def\tablelinecolor {black}       % Color de las líneas de las tablas
\def\titlecolor {black}           % Color de los títulos
\def\urlcolor {magenta}           % Color de los enlaces web (\url,\href)

% CONFIGURACIÓN DE FIGURAS
\def\defaultimagefolder {images/} % Carpeta raíz de las imágenes
\def\marginbottomimages {-0.2}    % Margen inferior figura [cm]
\def\marginfloatimages {-13.0}    % Margen sup. fig. float \insertimageleft/right [pt]
\def\margintopimages {0.0}        % Margen superior figura [cm]

% ANEXO, CITAS, REFERENCIAS
\def\apaciterefsep {9}            % Separación entre referencias [pt] {apacite}
\def\appendixindepobjnum {true}   % Anexo usa número objetos de forma independiente
\def\bibtexrefsep {9}             % Separación entre referencias [pt] {bibtex}
\def\donumrefsection {false}      % Sección de referencias numerada
\def\natbibrefsep {5}             % Separación entre referencias [pt] {natbib}
\def\sectionappendixlastchar {.}  % Carácter entre N° de la sec. de anexo y el título
\def\showappendixsectitle {false} % Muestra el título de la sec. de anexo en el informe
\def\stylecitereferences {bibtex} % Estilo citas y referencias (bibtex,apacite,natbib)
\def\twocolumnreferences {false}  % Referencias en dos columnas

% CONFIGURACIÓN DE LOS TÍTULOS
\def\anumsecaddtocounter {false}  % Fun. para insertar títulos <anum> aumenta contador
\def\bolditempto {true}           % Puntaje item en negrita
\def\fontsizesubsubtitle {\large} % Tamaño sub-subtítulos
\def\fontsizesubtitle {\Large}    % Tamaño subtítulos
\def\fontsizetitle {\huge}        % Tamaño títulos
\def\showdotontitles {false}      % Punto al final de cada N° de título/sub(sub)título
\def\stylesubsubtitle {\bfseries} % Estilo sub-subtítulos
\def\stylesubtitle {\bfseries}    % Estilo subtítulos
\def\styletitle {\bfseries}       % Estilo títulos

% OPCIONES DEL PDF COMPILADO
\def\addindextobookmarks {false}  % Añade el índice a los marcadores del pdf
\def\cfgbookmarksopenlevel {1}    % Nivel de los marcadores a mostrar (2: subsecciones)
\def\cfgpdfbookmarkopen {true}    % Expande los marcadores hasta el nivel configurado
\def\cfgpdfcenterwindow {true}    % Centra la ventana del lector al abrir el pdf
\def\cfgpdfcopyright {}           % Establece el copyright del documento
\def\cfgpdfdisplaydoctitle {true} % Muestra el título del informe como título del pdf
\def\cfgpdffitwindow {false}      % Ajusta la ventana del lector al tamaño del pdf
\def\cfgpdfmenubar {true}         % Muestra el menú del lector al abrir el pdf
\def\cfgpdfpagemode {OneColumn}   % Modo de página en el lector (OneColumn,SinglePage)
\def\cfgpdfpageview {FitH}        % Ajuste de p. (Fit,FitH,FitV,FitR,FitB,FitBH,FitBV)
\def\cfgpdfsecnumbookmarks {true} % Número de la sección en los marcadores del pdf
\def\cfgpdftoolbar {true}         % Muestra la barra de herramientas del lector pdf
\def\cfgshowbookmarkmenu {false}   % Muestra el menú de los marcadores al abrir el pdf
\def\pdfcompileversion {7}        % Version mínima del pdf a compilar (1.x)

% NOMBRE DE OBJETOS
\def\nameappendixsection{Anexos}  % Nombre de la sección de anexos/apéndices
\def\namereferences {Referencias} % Nombre de la sección de referencias
\def\nomltappendixsection {Anexo} % Etiqueta sección en anexo/apéndices
\def\nomltwfigure {Figura}        % Etiqueta leyenda de las figuras
\def\nomltwsrc {Código}           % Etiqueta leyenda del código fuente
\def\nomltwtable {Tabla}          % Etiqueta leyenda de las tablas
