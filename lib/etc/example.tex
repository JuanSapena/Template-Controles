% Template:     Template Controles LaTeX
% Documento:    Archivo de ejemplo
% Versión:      3.4.8 (23/01/2020)
% Codificación: UTF-8
%
% Autor: Pablo Pizarro R. @ppizarror
%        Facultad de Ciencias Físicas y Matemáticas.
%        Universidad de Chile.
%        pablo.pizarro@ing.uchile.cl, ppizarror.com
%
% Sitio web:    [https://latex.ppizarror.com/controles]
% Licencia MIT: [https://opensource.org/licenses/MIT]

\newquestionthemed{Pregunta 1}{Tema de la pregunta}

	% Párrafo
	\lipsum[114]
	
	\insertimage[\label{img:testimage}]{ejemplos/test-image.png}{scale=0.20}{Phasellus a ante. Donec et diam.}
	
	\begin{enumerate}[label=\alph*), format=\textbf]
		\itempto{1.0}{¿Cum sociis natoque penatibus et magnis dis parturient montes, nascetur ridiculus mus?}
		\itempto{3.0}{Lorem ipsum dolor sit amet, consectetuer adipiscing elit. ¿In hac habitasse platea dictumst?}
		\itempto{2.0}{Proin tempus nibh sit amet nisl:
			\begin{itemize}
				\item{Nunc vitae tortor.}
				\item{Nulla in ipsum.}
			\end{itemize}
		}
	\end{enumerate}

\newquestion{Pregunta 2}

	% Párrafo
	\lipsum[121]
	
	\begin{enumerate}[label=\alph*), format=\textbf]
		\itempto{1.0}{Vestibulum ante ipsum primis in faucibus orci luctus $\nabla x+\mho u = 0$}
		\itempto{1.0}{
			Proin ut est:
			\insertequationanum{
				\Gamma (\alpha \cdot \beta) + \prod_{i=0}^{\infty } \Phi (\alpha \cdot 2^{-i})
			}
		}
		\itempto{4.0}{\lipsum[101]}
	\end{enumerate}

\newquestionthemed{Pregunta 3}{Tema de la pregunta}

	\insertimageright{ejemplos/test-image-wrap}{0.35}{¿Ubi sum?}
	
	~ \lipsum[4]
	
	\begin{enumerate}[label=\alph*), format=\textbf]
		\itempto{2.0}{Aenean placerat. Ut imperdiet, enim sed gravida sollicitudin, felis odio placerat quam. ¿Ac pulvina elit purus eget enim?}
		\itempto{2.0}{In eget vitae, $a^n + b^n = c^n\quad\forall i \in \{\ldots\infty\}$}
		\itempto{1.0}{Nibh enim $\aleph_i + i = 0\quad \forall i \in [0,1)$}
		\itempto{1.0}{\lipsum[68]}
	\end{enumerate}